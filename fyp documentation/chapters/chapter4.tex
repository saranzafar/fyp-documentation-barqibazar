\section{Project Management Techniques}

\subsection{Gantt Chart}
The Gantt Chart is a widely used project management tool that provides a visual representation of project activities over time. It helps in organizing tasks, defining timelines, and monitoring progress throughout the development lifecycle. By breaking the project into smaller tasks and mapping them against a timeline, the Gantt Chart enables effective planning and coordination among team members.

One of the major advantages of using a Gantt Chart is its ability to represent task dependencies clearly. Certain tasks in a software project cannot begin until prerequisite activities are completed. The Gantt Chart highlights these dependencies, helping the project team avoid scheduling conflicts and delays. Additionally, it promotes accountability by clearly assigning responsibilities and deadlines.

In the Barqi Bazar project, the Gantt Chart has been used to plan and track activities such as requirements analysis, system design, implementation of POS and portal modules, testing, and documentation. This structured timeline ensured that all development phases were completed in a logical sequence and within the allocated timeframe.

\subsection{Critical Path Method (CPM)}
The Critical Path Method (CPM) is a project scheduling technique used to identify the most important sequence of tasks that directly impact the overall project completion time. The critical path represents the longest chain of dependent activities, and any delay in these tasks results in a delay of the entire project.

CPM is especially useful for managing software projects with multiple interdependent modules, such as POS operations, order management, rider assignment, and administrative portals. By identifying critical tasks early, project managers can allocate resources more effectively and focus attention on activities that require strict monitoring.

For the Barqi Bazar system, CPM was applied to identify essential milestones such as completion of core system architecture, POS workflow implementation, order processing logic, and integration of rider bidding and delivery assignment. This approach minimized risks, improved coordination, and ensured timely completion of the project.

\subsection{Benefits of Gantt Chart and CPM}
The combined use of Gantt Charts and the Critical Path Method provides a comprehensive approach to project planning and control. While the Gantt Chart offers a clear visual overview of task schedules, CPM focuses on identifying and managing time critical activities.

These techniques improve communication among team members and stakeholders by presenting project progress in a structured and understandable manner. They also help in early identification of potential delays, enabling corrective actions to be taken promptly.

In the context of Barqi Bazar, these project management techniques contributed to better organization, efficient resource utilization, and timely delivery of system components. Their use ensured that the project remained aligned with academic requirements and development goals.

\section{Work Breakdown Structure (WBS)}

\subsection{Introduction}
The Work Breakdown Structure (WBS) is a hierarchical decomposition of the project into smaller, manageable tasks. It provides a clear framework for organizing project activities and ensures that each phase of development is systematically planned and executed.

By dividing the project into well defined components, the WBS improves task clarity, simplifies progress tracking, and supports effective allocation of resources. It also helps identify dependencies between tasks and reduces the risk of overlooking critical activities.

\subsection{WBS for the Proposed System}
The Work Breakdown Structure for the Barqi Bazar system is presented below:

\begin{table}[H]
	\centering
	\caption{Work Breakdown Structure for Barqi Bazar}
	\label{table:wbs}
	\begin{tabular}{|p{0.15\linewidth}|p{0.75\linewidth}|}
		\hline
		\textbf{Level} & \textbf{Activity} \\ \hline
		1 & Project Proposal and Planning \\ \hline
		2 & Requirements Gathering and Analysis \\ \hline
		3 & System Design (Use Case, Sequence, Activity, ER Diagrams) \\ \hline
		4 & Implementation of POS, Portals, and Backend Services \\ \hline
		5 & Integration of Order, Rider, and Delivery Modules \\ \hline
		6 & System Testing and Validation \\ \hline
		7 & Documentation and Final Deployment \\ \hline
	\end{tabular}
\end{table}

Each level represents a major phase of the Barqi Bazar project. This structured breakdown enabled systematic execution and ensured that all functional and non functional requirements were addressed efficiently.

\section{Gantt Chart}

\subsection{Introduction}
A Gantt Chart provides a visual timeline of project tasks, showing their start and end dates, durations, and dependencies. It is an effective tool for tracking project progress and ensuring timely completion of activities.

For the Barqi Bazar project, the Gantt Chart was developed to plan stages such as analysis, design, development, testing, and documentation. It ensured smooth progression between phases and helped monitor adherence to the project schedule.

\subsection{Gantt Chart for the Proposed System}
\begin{figure}[H]
	\centering
	\includegraphics[width=1\textwidth]{figures/gantt-chart/gantt-chart.png}
	\caption{Gantt Chart for Barqi Bazar System}
	\label{fig:gantt_chart}
\end{figure}

The Gantt Chart illustrates the chronological flow of project activities and highlights task dependencies. It serves as a reference for monitoring milestones and ensuring that the project remains on schedule.

\subsection{Importance of the Gantt Chart}
The Gantt Chart plays a crucial role in coordinating project activities and improving collaboration among team members. By providing a clear visual representation of deadlines and progress, it aligns the team toward shared objectives.

Additionally, the chart helps identify potential scheduling issues early, allowing timely adjustments. In the Barqi Bazar project, it supported effective time management and contributed to the successful completion of development and documentation phases.
